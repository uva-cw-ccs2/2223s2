%\documentclass[handout]{beamer}
\documentclass[compress]{beamer}

\input{../resources/preamble}
\addbibresource{../resources/literature.bib}
\graphicspath{{../resources/pictures/}}

\title[Computational Communication Science 2]{\textbf{Computational Communication Science 2} \\Week 8 - Lecture\\ »Coding in an academic context«}
\author[Marthe Möller]{a.m.moller@uva.nl}\author[A. Marthe Möller]{Marthe Möller \\ ~ \\ \footnotesize{a.m.moller@uva.nl} \\}
\date{May 22, 2023}
\institute[Digital Society Minor, University of Amsterdam]{Digital Society Minor, University of Amsterdam}

\begin{document}
	
	\begin{frame}{}
		\titlepage
	\end{frame}
	
	\begin{frame}{Today}
		\begin{tiny}
			\tableofcontents
		\end{tiny}
	\end{frame}


\section{Recap}

\begin{frame}[fragile]{Recap} 
	
\begin{alertblock}{Techniques you now master:}
\begin{itemize}
	\item Text as data
	\item Recommender systems
	\item Supervised machine learning
\end{itemize}
\end{alertblock}

\begin{alertblock}{Today, we:}
\begin{itemize}
	\item Reflect on computational methods
	\item Discuss responsible coding
\end{itemize}
\end{alertblock}
\end{frame}


\section{A critical reflection}

\begin{frame}[fragile]{Relevance of Computational Techniques}
	
\begin{alertblock}{What can we use Python for?}	
\begin{itemize}
	\item Methodological advancement: Using Python in traditional methods. 
	\begin{itemize}
		\item Building a recommender system and using it in an experiment
	\end{itemize}
\end{itemize}
\end{alertblock}
\end{frame}

\begin{frame}[fragile]{Relevance of Computational Techniques}
	
	\begin{alertblock}{What can we use Python for?}	
		\begin{itemize}
			\item Analzying text as a goal: Studying text can teach us a lot about human behavior.  
			\begin{itemize}
				\item Studying what people discuss on cancer-related online platforms (\tiny{e.g., Sanders et al., 2020})
			\end{itemize}
		\end{itemize}
	\end{alertblock}
\end{frame}

\begin{frame}[fragile]{Relevance of Computational Techniques}
	
\begin{alertblock}{What can we use Python for?}	
\begin{itemize}
	\item Analyzing text as a means: Studying text can answer broader questions.
	\begin{itemize}
		\item Auomatically distinguish between reliable and unreliable online information about vaccines 
		by investigating what characterizes reliable and unreliable texts (\tiny{e.g., Meppelink et al., 2021})
	\end{itemize}
\end{itemize}
\end{alertblock}
\end{frame}



\begin{frame}{Three gaps in the development of CTAM}
	
\begin{alertblock}{Baden et al. (2022):}
\begin{itemize}
	\item Measurement validity versus technological properties.
\end{itemize}
\end{alertblock}	

% Social scientists measure complex constructs that can be measured in various ways. However, when using a computer, it is very hard to supervise for a human if the measuring happens in a valid way (unless it is very rule based), so validation is genreally very hard (e.g. it is hard to know WHY a classifier has a high accuracy score, for example). Operational validity versus predictive performance


% Consequences:
% Social scientists do not see their knowledge reflected in CTAM development
% Validation happens after the development of tools and on specific cases only. Because of this, the results are not fed back into the development and advancement of our theoretical understanding is hindered.
% Because social scientists are not included, they often have trouble finding helpful guidelines. Choices are often not substanted or explained.
\end{frame}


\begin{frame}{Three gaps in the development of CTAM}
	
\begin{alertblock}{Baden et al. (2022):}
\begin{itemize}
	\item Measurement validity versus technological properties.
	\item Focus on detecting single and simple constructs instead of multiple and complex ones.
\end{itemize}
\end{alertblock}

% For social scientists, meaning of text often depends on its context but computational tools focus on detecting elements on specific data levels
% As a consequences, scientists have to combine multiple tools, but as these are often developed as standalone (e.g., have their own assumptions about the data, are build in different languages) tools it is not always possible
% Difficult to get holistic insights
% The utility of CTAM depends on the number and kinds of constructs that scientists want to measure
\end{frame}


\begin{frame}{Three gaps in the development of CTAM}
	
\begin{alertblock}{Baden et al. (2022):}
\begin{itemize}
\item Measurement validity versus technological properties.
\item Focus on detecting single and simple constructs instead of multiple and complex ones.
\item Methods to study non-English texts underveloped. 
\end{itemize}
\end{alertblock}	

% Social science research is conducted in many different countries: so analysis of text is about various languages. This is not reflected in CTAM
% Many tools only available for English: so we gain knowledge about English texts, but not about others
% The properties of the English language have become hard-coded into computational logistic thinking and development (think about word-order and bi-grams)
% Disadvantage for resarchers studing non-English languages
\end{frame}



\begin{frame}{Consequences for scholars}
	
\begin{alertblock}{Baden et al. (2022):}
\begin{itemize}
	\item CTAM is applied to varying degrees in various disciplines.
	\item Blurring of theoretical concepts.
\end{itemize}
\end{alertblock}
	
	% The first point may even further strengthen the current problems: tools are being developed with a focus on one specific field, making them less usable for other fields/topics. In addition, some scholars willl receive more and more training whereas scholars in other fields stay behind.
	
	% The second point: if you start to use tools developed for one specific construct on other constructs, theoretical distinctions get lost.
	
\end{frame}


\begin{frame}{Consequences for society}
The imperfections of computational methods also have broader consequences.

% Baden et al identify problems that make comp methods less efficient for scholars. But we have to fix this because there are consequences beyond the scientific community!
\end{frame}


\begin{frame}{The harms of computing}
	
\begin{alertblock}{Bender et al. (2021):}
\begin{itemize}
	\item CO2 emissions associated with training and developing models.
	\item Cost of and access to hardware.
\end{itemize}
\end{alertblock}
"It is past time for researchers to prioritize energy efficiency and cost to reduce negative environmental impact and inequitable access to resources — both of which disproportionately affect people who are already in marginalized positions." (Bender et al., 2022, p. 613)
\end{frame}


\begin{frame}{About that input (again)}
	
\begin{alertblock}{Size does not garantuee diversity:}
\begin{itemize}
	\item Internet access is not evenly distributed
	% internet data overrepresents young people and those from developed countries
	\item Moderation causes marginalized populations to be less welcomed.
	% internet data only represents mainstream world views - the risk is that this creates a loop
	\item Changes in social views cannot be accounted for when retraining models is expensive. 
	%section 4.1 and 4.2 bender
\end{itemize}
\end{alertblock}
Bender et al., 2022
\end{frame}



\begin{frame}{Potential solutions}
	
\begin{alertblock}{Baden et al. (2022):}
\begin{itemize}
	\item Focus on validation.
	\item Combining strategies.
	\item Scholars as method developers instead of users.
	\item Define and operationalize constructs.
	\item Explain and argue for all the (pre-processing) that you do.
	\item Make the materials that you create publicly available.
\end{itemize}
\end{alertblock}

% More validation is needed. Combining strategies can make them more adequete to analyze complex constructs. Including scholars in the development of methods can solve current problems.
\end{frame}


\begin{frame}{Potential solutions}

Sustainable computing
\begin{alertblock}{Bender et al. (2021):}
\begin{itemize}
	\item Report training time.
	\item Add efficiency as an evaluation metrics.
	\item Reflect on who your models are valuable for.
	\item Invest in better training data.
\end{itemize}
\end{alertblock}
\end{frame}


\begin{frame}{Potential solutions}
This all starts with us!
\end{frame}


\section{Open Science}

\begin{frame}{Open Computational Science}
	
\begin{alertblock}{Sharing materials - advantages:}
\begin{itemize}
	\item Contributes to solving the problems outlined by scholars (e.g., Baden et al., 2022).
	\item Increases the transparency of your own projects.
	\item Increases the availability of and access to materials.
	\item Reduces the need to keep reinventing the wheel.
\end{itemize}
\end{alertblock}
\end{frame}

\begin{frame}{Open Computational Science}
	
\begin{alertblock}{Sharing materials - concerns:}
	\begin{itemize}
		\item Privacy of content creators (Hirschberg \& Manning, 2015).
		\item Who owns these data? Consider the role of platforms.
		\begin{itemize}
			\item Never make data publicily availble without ethical permission from the university!
		\end{itemize}
		\item Who owns the code? You! But:
		\begin{itemize}
			\item You do not own any packages or modules that you used, so you need to give proper credit.
		\end{itemize}
	\end{itemize}
\end{alertblock}
	% You cannot solve all the problems above, but you can take your own responsibility.
\end{frame}



\section{About Reproducible Code}

\begin{frame}{Reproducible code}
	
	Reproducible code is:
	\begin{itemize}
		\item Clear
		\item Readible
		\item Efficient
	\end{itemize}

% Easier to read but slightly slower code is better than faster but indecipherable code

	% Because:
	% Easier to understand
	% More efficient use of computer power
	% Easier to maintain, scale, debug
	% Requires less documentation
\end{frame}



\begin{frame}{Reproducible code}

PEP: Python Enhancement Proposal
	
\end{frame}


\begin{frame}{Writing code efficiently}
	
Pep8: APA for Python code \\
Goal: To improve the readibility of code. \\\
	
See: https://peps.python.org/pep-0008
	
%USe comments (https://peps.python.org/pep-0008/#comments)
%https://testdriven.io/blog/clean-code-python/)
\end{frame}

\begin{frame}
The Zen of Python (Peters, 2004): PEP20 \\\

Runs with command "import this"

%https://inventwithpython.com/blog/2018/08/17/the-zen-of-python-explained/
\end{frame}



\begin{frame}{Reproducible code}
	
A selection of The Zen of Python (Peters, 2004): \\\
	
Simple is better than complex. \\
Flat is better than nested. \\
Sparse is better than dense. \\\
Readability counts. \\\
Errors should never pass silently. Unless explicitly silenced. \\

Read all 19 rules on: https://peps.python.org/pep-0020/
	
% Avoid organizing your code into sub- sub- sub- categories

% Code that is spread over multiple lines each doing one thing are easier to read than code that is one line that does many different things - even though the latter may seem more 'advanced'

% It is better for a code to crash quickly than to silence an error and produce incorrect output
	
\end{frame}



\begin{frame}{Reproducible code}
	
How to achieve reproducible code?
	
\end{frame}

\begin{frame}{Open Computational Science}
1. Document your code
\end{frame}

\begin{frame}{Open Computational Science}
Documentation debt: "When we rely on ever larger datasets we risk incurring documentation debt, i.e. putting ourselves in a situation where the datasets are both undocumented and too large to document post hoc." (Bender et al., p. 615)
	
%Bender et al., p. 615, section 4.4
\end{frame}

\begin{frame}[fragile]{Source Acknowledgement}
	
\begin{lstlisting}		
"""
In your code, you can use """ or # to create  a comment explaining your code. For example, to say that you are using a scraper developed by A. Person (2015), which can be found on www.somewebsite.com
"""
\end{lstlisting}
\end{frame}


\begin{frame}[fragile]{Writing reproducible code}
	
2. Create functions  (instead of repeating code)
	
Not:
\begin{lstlisting}		
# Rerun this code three times, once for each name:
# Mike, Elsa, and Minna
		
print("[change name here] is cool!")
\end{lstlisting}
	
But:
\begin{lstlisting}		
names = ["Mike", "Elsa", "Minna"]
		
def cool_caller(name):
  print(name + " is cool!")
		
for name in names:
  cool_caller(name)
\end{lstlisting}
	
	
	%Increases readibility
	% You can call the same function again later on below in the code
	%Makes code easier to maintain (only one function needs to be updated)
\end{frame}

\begin{frame}{Writing reproducible code}
	
3. Eliminate unnecessary operations and break loops
	
Only interested in the first five elements of a list? Don't loop over the entire list!
\end{frame}


\begin{frame}{Writing reproducible code}
	
4. Avoid hard-coding values

Not:
"myfile.csv" or 50 within your script. \\\

But: \\\
OUTPUTFILE ="myfile.csv" \\\
MAXNUMBER=50
\end{frame}


\begin{frame}{Writing reproducible code}

5. Avoid defining unnecessary variables 

% Especially when defining variables again and again within loops, take them out of the loop!
\end{frame}


\begin{frame}{Writing reproducible code}
	
6. Use built-in functions and libraries 

And load all modules/packages at the start of your code
	
	%No need to reinvent the wheel, but:
	%Clearly indicate what packages  you are using (at the start of your code)
	%Give credit where credit is due, using comments
	
	%Keep track of the versions of libraries that you use in comments
	
\end{frame}

\begin{frame}{Writing reproducible code}
	
7. Use informative names for files and variables (check out PEP8!)
	
% By using _ and - strategically, you can use regex on filenames
\end{frame}


\begin{frame}{Writing reproducible code}
	
8. Keep lines of code fairly short

% Keeps your code clearer and easier to debug

\end{frame}


\begin{frame}{Writing reproducible code}
	
9. Position assignments close to their usage.

% Keeps it clear what you are doing and why
	
\end{frame}


\begin{frame}{Writing reproducible code}
	
10. Write code that makes sense to others, and it will make snese to the future you as well.
	
\end{frame}



\section{Looking Back and Ahead}

\begin{frame}{Looking back} 
	
You started with the basics (e.g., what is a list, how to save data, write a loop).\\
\begin{alertblock}{In two months, you learned:}
	\begin{itemize}
		\item How to read in data
		\item How to preprocess data
		\item How transform text into data that a computer can understand
		\item How to compare texts to provide a recommendation
		\item How to analyze text to classify it automatically		
	\end{itemize}
\end{alertblock}	
\end{frame}


\begin{frame}{Looking at the very near future} 
	
\begin{alertblock}{Two grades for this course remain:}
	\begin{itemize}
		\item The last MC-questions 
		\item The take-home exam 
	\end{itemize}
\end{alertblock}	
\end{frame}

\begin{frame}{Looking at the near future} 
\begin{alertblock}{	Final course of the minor: the research project!}
	\begin{itemize}
		\item You will combine your programming skills with your skills as a researcher
		\item Run a research project about ComScience using the materials you created for the group assignment in this course
		\item More information follows in the first meeting of the research project
	\end{itemize}
\end{alertblock}
\end{frame}


\begin{frame}{Looking at the future} 
	
	CCS-1 and CCS-2: An introduction to coding. You can continue to learn and work with Python.\\
	No course materials and instructors to help you out, but there are a lot of resources online!\\
\begin{alertblock}{Our tips:}
	\begin{itemize}
		\item Error message? Google is your best friend!
		\item Check out the documentation of any module that you use to learn how it works
		\item Check out pubs using the method you are interested in. Often, they publish their python scripts (e.g., Meppelink et al., 2021)
	\end{itemize}
\end{alertblock}	
\end{frame}


\begin{frame}{Looking at the future} 
	
When you use your computational skills for your thesis or any other project, you are part of the research community. You add value to your own work by making it accessible and understandable for others!
	
\end{frame}


\begin{frame}{Looking at the future} 
	
We taught you basic principles and techniques that underly frequently used methods. These techniques develop and change constantly - make sure that you stay updated on them!
	
% E.g., deep learning, developments in UML
	
\end{frame}



\begin{frame}{Looking at the future} 
	
We taught you how to drive, now you can go out and explore the world of computational (communication) science!
	
\end{frame}


\begin{frame}{Looking at the future} 
	
Thank you for the past weeks and enjoy the research project!
	
\end{frame}

\end{document}